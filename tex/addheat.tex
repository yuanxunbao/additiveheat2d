\documentclass[a4paper,12pt]{article}
\usepackage{amsmath}
\usepackage{amsfonts}
\usepackage{amssymb}
\usepackage{hyperref}
\usepackage{subfig}
\renewcommand{\figurename}{Fig.}
\renewcommand*{\figureautorefname}{Fig.}
\usepackage{graphicx}
\usepackage{float}
\usepackage{multirow}
\usepackage{placeins}
\usepackage{color}
\usepackage{array}
\usepackage{cancel}
\usepackage[margin=1in]{geometry}
%\usepackage[left=1.5in, right=1.5in]{geometry}

\usepackage[nameinlink,noabbrev]{cleveref}
\crefname{equation}{eq.}{eqs.} % force abbreviated forms for equation "names"
\Crefname{equation}{Eq.}{Eqs.}
\crefname{figure}{fig.}{figs.}
\Crefname{Figure}{Fig.}{Figs.}


%\usepackage{cmbright}
%\renewcommand{\familydefault}{\sfdefault}

\newcommand{\diff}{\mathrm{d}}
\newcommand{\V}[1]{\boldsymbol{#1}}
\newcommand{\B}[1]{\mathbf{#1}}
\newcommand{\myhat}[2]{\hat{#1}_{#2}}
\renewcommand*\arraystretch{1.5}
\renewcommand{\div}[1]{\nabla_{#1} \cdot}
\newcommand{\lapl}{\nabla^2}
\newcommand{\grad}[1]{\nabla_{#1}}
\newcommand{\curl}{\nabla \times}
\newcommand{\Tr}{\mathrm{Tr}}
\newcommand{\op}[1]{\mathcal{#1}}


\DeclareMathAlphabet\mathbfcal{OMS}{cmsy}{b}{n}


\title{Phase-field modeling of binary alloy solidification with coupled heat and solute diffusion}
\author{Yuanxun Bao}
\date{\today}


\begin{document}

\maketitle

\section{Model}

We consider a phase-field model for simulating quantitatively microstructural pattern formation in solidification of dilute binary alloys with coupled heat and solute diffusion. 
\begin{align}
\tau A^2(\theta) \frac{\partial \phi}{\partial t}  & =   W^2  \left\{ \div{} (A^2(\theta) \grad{} \phi) - \partial_x [ A(\theta) A'(\theta) \partial_y \phi ] + \partial_y [A(\theta) A'(\theta) \partial_x \phi ] \right\}, \\
 &  + \phi - \phi^3  - \lambda g'(\phi) \left( e^{u} - \frac{T-T_M}{m c_{\infty}} \right),  \nonumber \\
 \frac{\partial c}{\partial t }  & = \div{} [ D_l q(\phi) c \grad{} u - \vec{j}_{at} ], \\
 \frac{\partial T}{\partial t} &= \div{} (\kappa(\phi) \grad{} T) + \frac{1}{2} \frac{\partial \phi}{\partial t},
\end{align}
where 
\begin{align}
 u &= \ln \left(  \frac{2c/c_{\infty}}{1+k-(1-k) \phi } \right), \\
 q(\phi) &= \frac{1-\phi}{1+k - (1-k)\phi}, \\ 
 \vec{j}_{at} &= -\frac{c_{\infty}(1-k)W}{2\sqrt{2}} e^{u} \frac{\partial \phi}{ \partial t} \frac{\grad{} \phi}{|\grad{} \phi|}, \\
 \kappa(\phi) & = k_l  \frac{1-h(\phi)}{2} + k_s \frac{1+h(\phi)}{2},
\end{align}
where $h(\phi)$ is a monotonously increasing function of $\phi$ with $h(\pm 1) = \pm 1$, with $\phi = 1$ ($\phi = -1$) corresponding to solid (liquid). To model anisotropic growth of dendrites, we employ the function $A(\theta) = 1 + \epsilon \cos ( 4 \theta )$, where $\theta = \arctan(\phi_y / \phi_x)$.


\section{Numerical Discretization}
We use an equispaced grid in two spatial dimensions with $(x_i, y_j) = ( i \Delta x, j \Delta y)$, $\Delta x = \Delta y = L/N$ and $i,j = 0, \dots, N$. We define the quantities
\begin{align}
P & \equiv A^2(\theta)\phi_x - A(\theta) A'(\theta) \phi_y, \\ 
Q & \equiv A^2(\theta)\phi_y + A(\theta) A'(\theta) \phi_x,
\end{align}
and apply the standard finite difference method to discretize the phase-field equation
\begin{align}
\phi^{n+1}_{i,j} & = \phi^n_{i,j} \nonumber  \\ 
& + \frac{\Delta \bar{t}}{A^2(\theta_{i,j})}  \left\{ 
 \frac{1}{\Delta \bar{x}} \left(P_{i+\frac{1}{2},j} - P_{i-\frac{1}{2},j} \right) +   
  \frac{1}{\Delta \bar{x}} \left(Q_{i, j+ \frac{1}{2}} - Q_{i, j-\frac{1}{2}} \right)  \right\} \\
& + \phi_{i,j} - \phi_{i,j}^3  + q(\phi_{i,j}, c_{i,j}, T_{i,j}) \nonumber,
\end{align}
where
\begin{align}
(\phi_x)_{i+\frac{1}{2},j} &= \frac{\phi_{i+1,j} - \phi_{i,j}}{\Delta \bar{x}}, \\
(\phi_y)_{i+\frac{1}{2},j} &= \frac{1}{4\Delta \bar{x}} (\phi_{i,j}+\phi_{i+1,j}+\phi_{i,j+1}+\phi_{i+1,j+1}) - \frac{1}{4\Delta \bar{x}} (\phi_{i,j}+\phi_{i,j-1}+\phi_{i+1,j}+\phi_{i+1,j-1}),
\end{align}
and the values $(\phi_y)_{i,j+\frac{1}{2}}$ and $(\phi_x)_{i,j+\frac{1}{2}}$ are defined in a similar way.  We also need to evaluate $A(\theta)$ and $A'(\theta)$  in terms of $\phi_x$ and $\phi_y$. After some algebraic manipulation using trigonometric identities, we obtain
\begin{align}
A(\theta) & \equiv A[\phi] = 1 - 3\epsilon + 4 \epsilon \left(  \phi_x^4 + \phi_y^4  \right) / (\phi_x^2 + \phi_y^2)^2, \\
A'(\theta) & \equiv A'[\phi]= -16 \epsilon \phi_x \phi_y (\phi_x^2 - \phi_y^2) / (\phi_x^2 + \phi_y^2)^2.
\end{align}
The concentration equation can be efficiently discretized using a finite volume method
\begin{equation}
c^{n+1}_{i,j} = c^n_{i,j} - \frac{\Delta \bar{t}}{\Delta \bar{x}} \{ (\op{J}_R^n - \op{J}_L^n) + (\op{J}_T^n - \op{J}_B^n)\}
\end{equation}
where it is assumed $\Delta x = \Delta y$, and the term $\op{J}^n_R \equiv \op{J} \cdot \vec{i}$ denotes the normal component of the flux on the right edge of the finite volume box, where $\op{J}$ is given by 
\begin{equation}
\op{J} = -D_L q(\phi) c  \grad{} u - \vec{j}_{at}.
\end{equation} 
Taking the right edge of the finite volume box as an example, the quantities that enter $\op{J}_R^n$ are evaluated at $(x_{i+1/2},y_j)$ as follows:
\begin{align}
q(\phi^n_{i+1/2,j}) c^n_{i+1/2,j} &= q\left( \frac{\phi^n_{i+1,j} + \phi^n_{i,j}}{2} \right) \times \left( \frac{c^n_{i+1,j} + c^n_{i,j}}{2} \right), \\
\grad{} u \cdot \hat{i} \left|_{i+1/2,j} \right. =(u_x)_{i+1/2,j} &= \frac{e^u_{i+1,j} - e^u_{i,j}}{\Delta \bar{x} (e^u_{i+1,j}+e^u_{i,j})/2}, \\
\left[ e^u \frac{\partial \phi}{\partial t} \right]_{i+1/2,j}^{n+1} &= \left( \frac{e^{u(\phi^n_{i+1,j}, c^n_{i+1,j})} + e^{u(\phi^n_{i,j}, c^n_{i,j})} }{2} \right) \times  \left( \frac{\partial_t \phi^{n+1}_{i+1,j} + \partial_t \phi^{n+1}_{i,j}}{2} \right), \\
\frac{\partial \phi^n}{\partial x} \bigg|_{i+1/2,j}  &= \frac{\phi^n_{i+1,j}- \phi^n_{i,j}}{\Delta x} , \\
\frac{\partial \phi^n}{\partial y} \bigg|_{i+1/2,j}  &= \frac{\phi_{i,j+1}+\phi_{i+1,j+1} - (\phi_{i,j-1} + \phi_{i+1,j-1}) }{2(2\Delta y)} , \\
\hat{n}_x^R &= \frac{\frac{\partial \phi^n}{\partial x} \bigg|_{i+1/2,j}  }{ \left\{  \left( \frac{\partial \phi^n}{\partial x} |_{i+1/2,j} \right)^2  + \left( \frac{\partial \phi^n}{\partial y} |_{i+1/2,j} \right)^2 \right\}^{1/2} }.
\end{align}


% \bibliographystyle{unsrt}
% \bibliography{../library.bib}



%%%%%%%%%%%%%%%%%%%%%%%%%%%%%%%%%%%%%%%%%%%%%%%

\end{document}
